\newchapter{1.5cm}{Recommender systems}

This chapter introduces points of interest for RSs and their use at music streaming service Spotify.

\section{Historical perspective}

Recommender systems were created to provide users with recommendations on items they might enjoy or want. In the case of the RS above Grundy, it attempted to suggest books to read for users based on the perceived stereotype of the user. Grundy's perceived stereotype of the user changed as it served suggestions and received feedback or clarifications of wishes from the user. This recommender system was later classified as a user-modelling RS.\\

RSs have both evolved to work by using different methods to generate recommendations and their fields of application have grown immensely since the inception of Grundy.

\section{Recommender systems at Spotify}

A rich topology of RSs are used at Spotify, a widely used music streaming service.\\

Here, different methods such as collaborative filtering, machine learning and natural language processing is used to recommend songs and playlists to listeners. Recommendations are generated from the data that Spotify knows about its users and the music that they have already listened to, and have saved to their own playlists.












